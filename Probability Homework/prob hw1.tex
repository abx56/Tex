\documentclass{article}

\usepackage{fancyhdr}
\usepackage{extramarks}
\usepackage{amsmath}
\usepackage{amsthm}
\usepackage{amsfonts}
\usepackage{tikz}
\usepackage[plain]{algorithm}
\usepackage{algpseudocode}
\usepackage{amsmath}
\usepackage{verbatim}
%\usepackage{amssymb}

\usetikzlibrary{automata,positioning}

%
% Basic Document Settings
%

\topmargin=-0.45in
\evensidemargin=0in
\oddsidemargin=0in
\textwidth=6.5in
\textheight=9.0in
\headsep=0.25in

\linespread{1.1}

\pagestyle{fancy}
\lhead{\hmwkAuthorName}
\chead{\hmwkClass\ (\hmwkClassInstructor): \hmwkTitle}
\rhead{\firstxmark}
\lfoot{\lastxmark}
\cfoot{\thepage}

\renewcommand\headrulewidth{0.4pt}
\renewcommand\footrulewidth{0.4pt}

\setlength\parindent{0pt}

%
% Create Problem Sections
%

\newcommand{\enterProblemHeader}[1]{
    \nobreak\extramarks{}{Problem \arabic{#1} continued on next page\ldots}\nobreak{}
    \nobreak\extramarks{Problem \arabic{#1} (continued)}{Problem \arabic{#1} continued on next page\ldots}\nobreak{}
}

\newcommand{\exitProblemHeader}[1]{
    \nobreak\extramarks{Problem \arabic{#1} (continued)}{Problem \arabic{#1} continued on next page\ldots}\nobreak{}
    \stepcounter{#1}
    \nobreak\extramarks{Problem \arabic{#1}}{}\nobreak{}
}

\setcounter{secnumdepth}{0}
\newcounter{partCounter}
\newcounter{homeworkProblemCounter}
\setcounter{homeworkProblemCounter}{1}
\nobreak\extramarks{Problem \arabic{homeworkProblemCounter}}{}\nobreak{}

\newenvironment{homeworkProblem}{
    \section{Problem \arabic{homeworkProblemCounter}}
    \setcounter{partCounter}{1}
    \enterProblemHeader{homeworkProblemCounter}
}{
    \exitProblemHeader{homeworkProblemCounter}
}

%
% Homework Details
%   - Title
%   - Due date
%   - Class
%   - Section/Time
%   - Instructor
%   - Author
%

\newcommand{\hmwkTitle}{Homework\ \#1}
\newcommand{\hmwkDueDate}{September 19, 2017}
\newcommand{\hmwkClass}{PROBABILITY}
%\newcommand{\hmwkClassTime}{Section 005}
\newcommand{\hmwkClassInstructor}{Professor Regina}
\newcommand{\hmwkAuthorName}{Fan Yang}

%
% Title Page
%

\title{
    \vspace{2in}
    \textmd{\textbf{\hmwkClass:\ \hmwkTitle}}\\
    \normalsize\vspace{0.1in}\small{Due\ on\ \hmwkDueDate\ at 3:10pm}\\
    \vspace{0.1in}\large{\textit{\hmwkClassInstructor}}
    \vspace{3in}
}

\author{\textbf{\hmwkAuthorName}}
\date{}

\renewcommand{\part}[1]{\textbf{\large Part \Alph{partCounter}}\stepcounter{partCounter}\\}

%
% Various Helper Commands
%

% Useful for algorithms
\newcommand{\alg}[1]{\textsc{\bfseries \footnotesize #1}}

% For derivatives
\newcommand{\deriv}[1]{\frac{\mathrm{d}}{\mathrm{d}x} (#1)}

% For partial derivatives
\newcommand{\pderiv}[2]{\frac{\partial}{\partial #1} (#2)}

% Integral dx
\newcommand{\dx}{\mathrm{d}x}

% Alias for the Solution section header
\newcommand{\solution}{\textbf{\large Solution}}

% Probability commands: Expectation, Variance, Covariance, Bias
\newcommand{\E}{\mathrm{E}}
\newcommand{\Var}{\mathrm{Var}}
\newcommand{\Cov}{\mathrm{Cov}}
\newcommand{\Bias}{\mathrm{Bias}}

\begin{document}

\maketitle

\pagebreak

\begin{homeworkProblem}
    A coin is tossed three times and the sequence of heads and tails is recorded.

    \begin{enumerate}
        \item[(a)] \ List the sample space
        \item[(b)] \ List the elements that make up the following events: 1) A = at least two heads;
2) B = the first two tosses are heads; 3) C = the last toss is a tail.
        \item[(c)] \ List the elements of the following events: $1) A^c ; 2) A \cap B ; 3) A\cup C.$
    \end{enumerate}

    \textbf{Solution}

    We solve each solution algebraically to determine a possible constant
    \(c\).
    \\

    \textbf{(a)}
    denote H for heads and T for tails.
    \[
        \begin{split}
            S=
                    \{&(H,H,H),(H,H,T),(H,T,H),(H,T,T)
            \\
            &(T,H,H),(T,H,T),(T,T,H),(T,T,T) \}
            \\
        \end{split}
    \]

    \textbf{(b)}

    \[
        \begin{split}
            1)& A=\{(H,H,T),(H,T,H),(T,H,H),(H,H,H)\}
            \\
            &B= \{(H,H,T),(H,H,H)\}
            \\
            &C=\{(H,H,T),(H,T,T),(T,H,T),(T,T,T)\}
        \end{split}
    \]
    \textbf{(c)}

    \[
        \begin{split}
            1)&~A^c=\{(H,T,T),(T,H,T),(T,T,H),(T,T,T)\}
            \\
            2)&~A \cap B = \{(H,H,T),(H,H,H)\}=B
            \\
            3)&~A \cup C = \{ (H,H,T)\}
        \end{split}
    \]

\end{homeworkProblem}

\begin{homeworkProblem}

    \textbf{(a)}
    \\denote H for heads and T for tails.
    \[
        \begin{split}
            &number~of~sample~space=\left( \begin{array}{c}52\\5\end{array} \right)
            \\
            &number~of~choices~of~ suits=\left(\begin{array}{c}4\\1\end{array}\right)
            \\
            &P=\frac{ \# of suits}{\# of sample~space}\\
            &=\frac{\left(\begin{array}{c}4\\1\end{array}\right)}{\left( \begin{array}{c}52\\5\end{array} \right)}\\
            &=\frac{1}{649740}
            \\
        \end{split}
    \]

    \textbf{(b)}

    \[
        \begin{split}
            & list~as~(A2345)...(10JQKA),we~know~there~are~10~choices~of~value\\
            &there~are~4~choices~of~suit\\
            \\
            &P= \frac{10*4}{\left( \begin{array}{c}52\\5\end{array} \right)}\\
            &~~=\frac{1}{64974}
        \end{split}
    \]
    \textbf{(c)}

    \[
        \begin{split}
            &there~are~13~choices~of~value\\
            &there~are~4~choices~of~suit\\
            \\
            &P= \frac{13*4}{\left( \begin{array}{c}52\\5\end{array} \right)}\\
            &~~=\frac{1}{49980}
        \end{split}
    \]
    \textbf{(d)}

    \[
        \begin{split}
            &\text{since the 5 cards have same suits, they should not have same value}\\
            &there~are~\left( \begin{array}{c}13\\5\end{array} \right)~choices~of~value\\
            &there~are~4~choices~of~suit\\
            \\
            &P= \frac{\left( \begin{array}{c}13\\5\end{array} \right)*4}{\left( \begin{array}{c}52\\5\end{array} \right)}\\
            &~~=\frac{33}{16660}
        \end{split}
    \]

    \textbf{(e)}

    \[
        \begin{split}
            &\text{after chose value for the 3 cards ,there are only 12 choices for the other 2 cards. }\\
            &there~are~13*12*11~choices~of~value~for~no~repeated~value~
            of~the~other~2~cards.\\
            &in~this~situation~there~are~4*4*4~choices~of~suit.\\
            &besides~there~are~13*12~choices~of~value~for~a~pair.\\
            &in~this~situation~there~are~4*4*3~choices~of~suit.\\
            &P= \frac{(13*12*11)*(4*4*4)+(13*12)*(4*4*3)}{\left( \begin{array}{c}52\\5\end{array} \right)}\\
            &~~=\frac{188}{4165}
        \end{split}
    \]
    \textbf{(f)}

    \[
        \begin{split}
            &\text{there are 13*12 choices of value.}\\
            &there~are~\frac{4}{2} * \frac{4}{2}~choices~of~suit.\\
            &P= \frac{(13*12)*\frac{4}{2}*\frac{4}{2}}{\left( \begin{array}{c}52\\5\end{array} \right)}\\
            &~~=\frac{9}{4165}
        \end{split}
    \]

\end{homeworkProblem}

\begin{homeworkProblem}

    \textbf{(a)}
    \[
        \begin{split}
    &\text{there are 16 choices for women president. And there are 48 choices for president}\\
    &so~P(E)=\frac{16}{48}=\frac{1}{3}\\
        &\text{there are 32 choices for men vice president.
        And there are 48 choices for vice president}\\
        &so~P(F)=\frac{32}{48}
        ~=\frac{2}{3}\\
        &\text{there are}16*15+32*31\text{ choices for president of same sex}\\
        &\text{there are 48*47 choices for president}\\
        &so~P(G)=\frac{16*15+32*31}{48*47}\\
        &=\frac{77}{141}\\
        \end{split}
    \]

    \textbf{(b)}

    \[
        \begin{split}
            & E\cap F \text{represent president is woman and vice president is man}\\
            &\text{there are}16*32\text{choices for this situation}\\
            & E\cup F \text{represent president is woman or vice president is man}\\
            &\text{there are}16*32+16*15+32*31\text{choices for this situation}\\
            & E\cap F\cap G \text{does not make sense}\\
            &\text{there are}0\text{choices for this situation}\\
            &P(E\cap F)= \frac{16*32}{48*47}=\frac{32}{141}\\
            &P(E\cup F)= \frac{16*32+16*15+32*31}{48*47}=\frac{109}{141}\\
            &P(E\cap F\cap G)=0
        \end{split}
    \]
    \textbf{(c)}

    \[
        \begin{split}
        &P(G|E\cup F)=\frac{P(G\cap (E\cup F))}{P(E\cup F)}\\
        &G\cap (E\cup F)\text{represent two president are of same sex}=G\\
        &therefore~P(G|E\cup F)=\frac{P(G)}{P(E\cup F)}
        =\frac{\frac{77}{141}}{\frac{109}{141}}=\frac{77}{109}\\
        \end{split}
    \]

\end{homeworkProblem}

\begin{homeworkProblem}
    \[
        \begin{split}
    &\text{consider the event as inserting four adjacent aces into 48 shuffled cards}\\
    &\text{\# of order of adjacent aces are 4*3*2*1}\\
    &\text{\# of choices of inserting the four adjacent are} \left(\begin{array}{c}49\\1\end{array}\right) \\
    &therefore~P=\frac{4*3*2*1}{49}=\frac{24}{49}
        \end{split}
    \]
\end{homeworkProblem}

\begin{homeworkProblem}

    \textbf{(a)}
    \[
        \begin{split}
    &\text{\# of the sample space}=
        \left(\begin{array}{c}60\\30\end{array}\right)\\
        &\text{there are} \left(\begin{array}{c}60-5\\30\end{array}\right) \text{choices for this situation.}\\
        &so~P=\frac{\left(\begin{array}{c}55\\30\end{array}\right)}
        {\left(\begin{array}{c}60\\30\end{array}\right)}\\
        &=\frac{117}{4484}\\
        \end{split}
    \]

    \textbf{(b)}

    \[
        \begin{split}
            &\text{\# of the sample space}=
        \left(\begin{array}{c}60\\30\end{array}\right)\\
        &\text{there are} \left(\begin{array}{c}5\\4\end{array}\right)*
         \left(\begin{array}{c}60-5\\30-4\end{array}\right)
         \text{choices for this situation.}\\
        &so~P=\frac{\left(\begin{array}{c}5\\4\end{array}\right)*
        \left(\begin{array}{c}60-5\\30-4\end{array}\right)}
        {\left(\begin{array}{c}60\\30\end{array}\right)}\\
        &=\frac{675}{4484}\\
        \end{split}
    \]

    \textbf{(c)}

    \[
        \begin{split}
            &\text{\# of the sample space}=
        \left(\begin{array}{c}60\\30\end{array}\right)\\
        &\text{there are} \left(\begin{array}{c}60-5\\30-1\end{array}\right)
         \text{choices for this situation.}\\
        &so~P=\frac{\left(\begin{array}{c}60-5\\30-1\end{array}\right)}
        {\left(\begin{array}{c}60\\30\end{array}\right)}\\
        &=\frac{135}{4484}\\
        \end{split}
    \]

\end{homeworkProblem}

\begin{homeworkProblem}

    \textbf{(a)}
    \[
        \begin{split}
        &\text{this event could be either first up then down or first down then up}\\
        &so~P=p*(1-p)+(1-p)*p\\
        &=2p(1-p)
        \end{split}
    \]

    \textbf{(b)}

    \[
        \begin{split}
        &\text{there should be 1 day down and 2 days up}\\
        &so~P=\left(\begin{array}{c}3\\1\end{array}\right)*
        p^2 *(1-p)\\
        &=3p^2 (1-p)\\
        \end{split}
    \]

    \textbf{(c)}

    \[
        \begin{split}
        & \text{denote E for price increasing on the first day,then}~P(E)=p\\
        & \text{denote F for after three days price increased by 1,then}~P(F)=3p^2 (1-p)\\
        & \text{the probability that the first day price goes up and three days later price still increased by 1 should be}\\
        &P(E\cap F)=p*\left(\begin{array}{c}2\\1\end{array}\right)*p*(1-p)\\
        &=2p^2 (1-p)\\
        &P(E|F)=\frac{P(E\cap F)}{P(F)}\\
        &=\frac{2p^2 (1-p)}{3p^2 (1-p)}\\
        &=\frac{2}{3}\\
        \end{split}
    \]

\end{homeworkProblem}


\begin{homeworkProblem}



    \[
        \begin{split}
        &\text{this event could be either first up then down or first down then up}\\
        &so~P=p*(1-p)+(1-p)*p\\
        &=2p(1-p)
        \end{split}
    \]

\end{homeworkProblem}

\end{document}
